\documentclass[titlepage, a4paper]{article}
\usepackage{hyperref}
\usepackage[utf8]{inputenc}
\usepackage[english]{babel}


\title{
	Artificial Poker Player \\
	IT3105 \\
}
\author{
	Nordmoen, Jørgen H. \\
	Østensen, Trond
}

\date{\today}

\begin{document}
\pagenumbering{RomanType}
\maketitle

\begin{abstract}\label{abstract}
In this text we will explain our implementation of an A.I. Poker player in the course
IT3105. We will explain each phase of the project, what choices we made and our decisions
that have effected the outcome in each phase. We will include some results of the different
phases and wrap up with our thoughts on the implementation and possible future work.
\end{abstract}

\newpage
\tableofcontents

\pagenumbering{arabic}

\section{Basic Structure}\label{basic}
Since we chose to go with Java instead of Python as our programming language in this course
our structure might not be as simple and straight forward because of limitations or design
off Java.

We chose to utilize Java because we both recognized the advantages of having the possibility
of threading our code. Python does not support true multithreading\footnote{\url{http://wiki.python.org/moin/GlobalInterpreterLock}} and since we both have good experience with Java we chose to go with that.
Of course multithreading is not everything and the speed afforded to us by Java is quite
apparent compared to Python. This choice however did mean that we could not use the code that
was given to us in the course which meant we had some catching up to do in the starting phases.

\subsection{Poker playing}\label{poker playing}
We started out by implementing the basic of any card game, the cards, the deck and power rating.
The later was much inspired by the code that we had gotten from the course, but implemented as
a class which could be compared to others with most of the Java interfaces\footnote{\url{http://docs.oracle.com/javase/7/docs/api/java/lang/Comparable.html}}. We spent quite
some time getting this class right and have had some strange problems from time to time, but
we ended up with something that can really do its job. 

Most of the work of this class is
just to determine the best rank that a player can get from either an array of cards or
a poker hand and some community cards and be able to compare it self with other power ratings.
Looking through this code there might be some odd bits and pieces that stick out, the
lazy evaluation is probably a bit of a surprise. The reason for this choice is that we quite
quickly realized that when we compared power ratings in the roll-out simulation most of the
time we only need the rank it self because most evaluations of \textit{compareTo} will end then and
there and doesn't need to do the costly evaluation of determining which cards should be
kept and which kickers to use.

To implement the game it self we used a class called \textit{PokerMaster} which deals with
all the poker playing in our code. To enable it to support multiple phases of poker players
we designed an interface which allows the \textit{PokerMaster} to interact with all of the
phases in a uniform manner. This interface, which can be found in \textit{PokerPlayer},
supports all the methods that we needed in the later phases, but have seen several revisions
before it got to this stage. This interface is then backed up by an abstract class which
implements some of the tasks which every phase needed.

\textit{AbstractPokerPlayer} does
most of the work regarding chip count and make sures every phase pays what they need in
order to participate further in a game. It also makes sure that blinds are payed when
that is needed. When it comes to paying to the table we decided, for simplicity, that we
would allow each phase to go negative. This has some ramifications for the game, because
it means that there could potentially be much more chips involved than we intended at the
start, but that have not been a huge problem as we force some of the phases to bet less
when they are out of chips.

To make it easier for us on determining which player goes when we created the
\textit{PokerTable} class which contains some methods for keeping track of the
blinds at the table and also which player is big and small blind. The \textit{PokerMaster}
uses this class to retrieve the small and big blind and also who to deal cards from.

To give each player cards we designed the \textit{PokerHand} class which is just a helper
class to enable us to compare hands, easily create power ratings and have a simple way
of passing two cards around. It can compare it self to other hands, but this is not
used much since we mostly deal in power ratings.

Since we share the \textit{PokerMaster} code between the different phases there
is not much else to say about this code in regard to the phases. We have some
opponent modeling code which is only used by phase 3, but we will come back to that
in section \ref{opponent modeling}.

\subsection{Roll-out simulation}\label{roll-out simulation}
As the code progressed we started looking at roll-out simulation in phase 2 since
that was one of the big challenges in this project.

We started by looking at what we needed to simulate a complete poker game and came
to the conclusion that we should make a separate poker simulator from \textit{PokerMaster}.
Since the roll-out simulation is not dependant on any players playing the game
or any notion of betting and calling we decided that the \textit{PreFlopMaster}
would only deal cards and compare them at the end of the game to speed up the 
calculation. The \textit{PreFlopMaster} gets the hand we want to gather statistics
about and the number of players to play against. It then deals out cards to the
"other" players and begins to deal the flop, turn and river. Once this is done
it declares a winner and updates its statistics for that hand. It then goes on to
simulate X amount of simulations, decidable on the command-line, with the given cards
and a deck to play with. Once all the simulations are done it returns a
\textit{TestResult} instance with the given wins, ties and loses and a ratio.

As we mentioned before we decided to go with Java because we realized that this roll-out
simulation could easily be threaded to increase performance and so we did. We
start by creating all the possible poker hands according to the hole-card equivalence
classes\footnote{\url{www.idi.ntnu.no/emner/it3105/lectures/ai-poker-players.pdf}}.
Then we split these poker hands into lists of equal lengths with the number of
lists equal to the number of processors reported by Java\footnote{\url{http://docs.oracle.com/javase/7/docs/api/java/lang/Runtime.html\#availableProcessors\%28\%29}}.
We then create \textit{RolloutSimulators} which get a list each which then for each
card create a new \textit{PreFlopMaster} and simulates from 2 to 10 players and
writes the results to a file. From what we can see this threading makes it possible
to do quite a lot more simulations than without the threading and the factor
that dominates the time is the shuffling of the deck which is done 7 times for
each simulation.

From our limited experiments this allows us to do 100 000 simulations per hand with
2 to 10 players in about 15 minutes on a 4 core Intel i7 with HyperThreading.


\end{document}
