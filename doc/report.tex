\documentclass[titlepage, a4paper]{article}
\usepackage{hyperref}
\usepackage[utf8]{inputenc}
\usepackage[english]{babel}


\title{
	Artificial Poker Player \\
	IT3105 \\
}
\author{
	Nordmoen, Jørgen H. \\
	Østensen, Trond
}

\date{\today}

\begin{document}
\pagenumbering{RomanType}
\maketitle

\begin{abstract}\label{abstract}
In this text we will explain our implementation of an A.I. Poker player in the course
IT3105. We will explain each phase of the project, what choices we made and our decisions
that have effected the outcome in each phase. We will include some results of the different
phases and wrap up with our thoughts on the implementation and possible future work.
\end{abstract}

\newpage
\tableofcontents

\pagenumbering{arabic}

\section{Basic Structure}\label{basic}
Since we chose to go with Java instead of Python as our programming language in this course
our structure might not be as simple and straight forward because of limitations or design
off Java.

We chose to utilize Java because we both recognized the advantages of having the possibility
of threading our code. Python does not support true multithreading\footnote{\url{http://wiki.python.org/moin/GlobalInterpreterLock}} and since we both have good experience with Java we chose to go with that.
Of course multithreading is not everything and the speed afforded to us by Java is quite
apparent compared to Python. This choice however did mean that we could not use the code that
was given to us in the course which meant we had some catching up to do in the starting phases.

\subsection{Poker playing}\label{poker playing}
We started out by implementing the basic of any card game, the cards, the deck and power rating.
The later was much inspired by the code that we had gotten from the course, but implemented as
a class which could be compared to others with most of the Java interfaces\footnote{\url{http://docs.oracle.com/javase/7/docs/api/java/lang/Comparable.html}}. We spent quite
some time getting this class right and have had some strange problems from time to time, but
we ended up with something that can really do its job. 

Most of the work of this class is
just to determine the best rank that a player can get from either an array of cards or
a poker hand and some community cards and be able to compare it self with other power ratings.
Looking through this code there might be some odd bits and pieces that stick out, the
lazy evaluation is probably a bit of a surprise. The reason for this choice is that we quite
quickly realized that when we compared power ratings in the roll-out simulation most of the
time we only need the rank it self because most evaluations of \textit{compareTo} will end then and
there and doesn't need to do the costly evaluation of determining which cards should be
kept and which kickers to use.

To implement the game it self we used a class called \textit{PokerMaster} which deals with
all the poker playing in our code. To enable it to support multiple phases of poker players
we designed an interface which allows the \textit{PokerMaster} to interact with all of the
phases in a uniform manner. This interface, which can be found in \textit{PokerPlayer},
supports all the methods that we needed in the later phases, but have seen several revisions
before it got to this stage. This interface is then backed up by an abstract class which
implements some of the tasks which every phase needed.

\textit{AbstractPokerPlayer} does
most of the work regarding chip count and make sures every phase pays what they need in
order to participate further in a game. It also makes sure that blinds are payed when
that is needed. When it comes to paying to the table we decided, for simplicity, that we
would allow each phase to go negative. This has some ramifications for the game, because
it means that there could potentially be much more chips involved than we intended at the
start, but that have not been a huge problem as we force some of the phases to bet less
when they are out of chips.

To make it easier for us on determining which player goes when we created the
\textit{PokerTable} class which contains some methods for keeping track of the
blinds at the table and also which player is big and small blind. The \textit{PokerMaster}
uses this class to retrieve the small and big blind and also who to deal cards from.

To give each player cards we designed the \textit{PokerHand} class which is just a helper
class to enable us to compare hands, easily create power ratings and have a simple way
of passing two cards around. It can compare it self to other hands, but this is not
used much since we mostly deal in power ratings.

Since we share the \textit{PokerMaster} code between the different phases there
is not much else to say about this code in regard to the phases. We have some
opponent modeling code which is only used by phase 3, but we will come back to that
in section \ref{opponent modelling}.

\section{Roll-out simulation}\label{roll-out simulation}
As the code progressed we started looking at roll-out simulation in phase 2 since
that was one of the big challenges in this project.

We started by looking at what we needed to simulate a complete poker game and came
to the conclusion that we should make a separate poker simulator from \textit{PokerMaster}.
Since the roll-out simulation is not dependant on any players playing the game
or any notion of betting and calling we decided that the \textit{PreFlopMaster}
would only deal cards and compare them at the end of the game to speed up the 
calculation. The \textit{PreFlopMaster} gets the hand we want to gather statistics
about and the number of players to play against. It then deals out cards to the
"other" players and begins to deal the flop, turn and river. Once this is done
it declares a winner and updates its statistics for that hand. It then goes on to
simulate X amount of simulations, decidable on the command-line, with the given cards
and a deck to play with. Once all the simulations are done it returns a
\textit{TestResult} instance with the given wins, ties and loses and a ratio.

As we mentioned before we decided to go with Java because we realized that this roll-out
simulation could easily be threaded to increase performance and so we did. We
start by creating all the possible poker hands according to the hole-card equivalence
classes\footnote{\url{www.idi.ntnu.no/emner/it3105/lectures/ai-poker-players.pdf}}.
Then we split these poker hands into lists of equal lengths with the number of
lists equal to the number of processors reported by Java\footnote{\url{http://docs.oracle.com/javase/7/docs/api/java/lang/Runtime.html\#availableProcessors\%28\%29}}.
We then create \textit{RolloutSimulators} which get a list each which then for each
card create a new \textit{PreFlopMaster} and simulates from 2 to 10 players and
writes the results to a file. From what we can see this threading makes it possible
to do quite a lot more simulations than without the threading and the factor
that dominates the time is the shuffling of the deck which is done 7 times for
each simulation.

From our limited experiments this allows us to do 100 000 simulations per hand with
2 to 10 players in about 15 minutes on a 4 core Intel i7 with HyperThreading.

\section{Hand strength}\label{hand strength}
As the roll-out simulation had started to work at this point we started looking at
the phase 2 player and started to implement \textit{Hand strength} calculation.
Hand strength is used after the flop in order for the phase 2 player to determine if
the current hand can be beaten by other hands. This is done by comparing the current
hand and the current community cards to all other combination of hands that is possible
to achieve without the cards in the hand or on the table. 

This to is a simulation that can be done in parallel and we did this to. 

We have implemented two different
calculations the one described in our handed out materials which does not take into
account the possible turn and river cards and one which does. The first one is simply
implemented by creating a fresh deck, remove the cards on the players hand and the
given community cards and then create all permutations of poker hands that is possible
to get and then compare each with the players hand to see how many times the player
wins, ties and loses. To run this in parallel we just split the poker hand permutations
into lists as before and calculated all the wins, ties and loses in parallel before
we collected all the statistics together and return a hand strength ratio.

For the second implementation we deal out all possible turn and river cards, again
except for the players cards and the community cards, and create all possible poker
hand permutations and do the same calculation as above. The result of this calculation
is the lowest possible score that a pair of cards can give, meaning it is much more
accurate than the above calculation.

A flaw in the first hand strength calculation above is that it will most likely
over estimate the strength of a hand after the flop. For instance lets say the player
we are looking at are sitting with two aces on h*s hand and one ace is on the table
together with a 7 and a 10 of some suits. Since there can not be a straight present
and lets say for this example that only the 7 and 10 are of the same suit, while
the ace on the table is of a different suit there can not be a flush either. So the
simple hand strength calculation reports that the strength is 1.0. This might be
a basis for a player to bet because he/she thinks that they have the best cards and
nothing can beat them. But the flaw is that the turn could be a card with the same suit
as the 7 and 10 and then someone might have a flush or even a straight might begin to
appear. The second implementation would account for this and give the lowest ratio
that could possibly appear which would be an underestimate, but it could be
much more accurate because it doesn't overestimate.

\section{Opponent modelling}\label{opponent modelling}
When it comes to opponent modelling most of the interesting features is in the
contexts that we use to define a situation in the poker game that we think might
come up again.

Each context gets added by the \textit{PokerMaster} after each
player make a move. When a game reaches showdown the \textit{PokerMaster} informs
the opponent modeller that a showdown has been reached and gives the modeller the
players that have reached the showdown. The modeller then looks through all the 
contexts that have been added this round and selects those that have the players
in the showdown in them. It then calculates their hand strength at each context
and add these contexts to the pool of all seen contexts. When a player later
gives the opponent modeller a context it checks the list to see if that context
has been seen before and if it has it calculates an average hand strength and
a standard deviation\footnote{\url{http://mathworld.wolfram.com/StandardDeviation.html}}
and returns this to the player. The player can then use this information
to assess the potential hand of an opponent and select an action based on this.

\subsection{Context model}\label{context model}
In our setup we have chosen to take the below factors into account when creating
a context.

\begin{itemize}
	\item The player
	\item The action the player took. I.e. Bet or call
	\item The current round. E.g. Pre-flop, post-flop etc.
	\item The amount of players still playing. We chose to split this into
		three categories, "FEW", "NORMAL" and "MANY". Where few players
		are anything below 1/3, normal is below 3/4 and many is above that.
	\item The pot odds size which is split into three again "SMALL", "MEDIUM"
		and "LARGE". Where small is pot odds above 0.45, medium is above
		0.25 and large is below that.
\end{itemize}

From testing this seems to be a good fit, because we get a few different context, but
we also seem them again quite often. For the most part this is a good fit, we could
extend the number of different possibilities for the amount of players and pot odds, but
that has not been a big problem up until now.

\section{Player behavior}\label{player behavior}
In this section we will describe each player phase in a bit more detail than what
we have seen up until now. As we said in section \ref{poker playing} some of the
code is shared between all phases, but for the betting decision everything is
up to the implementation.

\subsection{Phase 1}\label{phase 1 player}
The first phase player is the simplest player of any of the phases. It has no 
information about the start of the game or opponent modelling and bases most
of its decisions on chance and power rating. Before the flop this player
has nothing to base its decision on so it just uses chance to decide whether
to bet, call or fold. Most of the time the player should just try to call to
get to the flop, but sometimes it will bet and sometimes it will fold with the folding
as the least possible outcome.

The intuition here is much the same as for a novice poker player which has little
grasp of the game. Before the flop it is hard to estimate what sort of chances
one has and thus we just leave it up to chance, but with a preference for calling.
This is done because none of the other player will likely bet a lot at this stage
and a flop will give us so much more information to continue so seeing it is a 
smart choice to make.

If this phase 1 player get to see the flop be do things a bit different. Now we
can use our power rating as a basis for telling us if the hand is worth something.
We still utilize random chance, but this time we lay more emphasis on the value
of the hand. What we do is to see what kind of hand we have, if it is a good hand
our value will be higher than random chance and we do not fold, but we retain this
randomness to sometime bluff and throwaway somewhat good cards in order to not
play static all the time. Then we see if we have a very good hand, if so we bet and
if not we just call to see if we can improve or have the other players fold.


\subsection{Phase 2}\label{phase 2 player}
The second phase player is quite a bit more sophisticated than the first one. This
is due to the new information available in this phase. In phase 2 the player has
access to both roll-out simulation(\ref{roll-out simulation}) and hand 
strength(\ref{hand strength}) calculation. With all this information the player
can take a much more qualified decision about what to do. 

With the roll-out simulation
the player has some statistics to tell whether or not the hole-cards is worth
continuing with before the flop. This information enables the player to know
whether or not he/she should continue to see the flop or fold the cards. This
enables strategies based on folding before the flop because the player has
an idea of how strong the cards are or have the potential to become later
in the game. 

The hand strength calculation is used after the flop. Because there are now cards
on the table we need to know if we have gotten anything better than just the two
hole-cards. The hand strength calculation helps with this because it calculates,
as mentioned above, the strength of the current hole-cards and community cards
compared to all the other possible hole-cards. This helps the player decide if
he/she should continue to play or, if there is a high chance anyone else has better
cards, fold.

In our code, before the flop, we consult the roll-out simulation statistics to see
if our cards are any good. If the statistics is over a certain threshold we decide
to bet. Should the statistics not be good enough to bet we check to see if the
hole-cards have the potential to become something after the flop and call. If this
is not the case we fold as this means the hole-cards have little potential for improvement.

After the flop we move over to combining hand strength, a bit of chance and a
configurable aggressiveness. We have again used chance to incorporate some element
of bluffing. If the random dice throw is large enough and the hand strength is weak
we fold, but if the dice throw had been smaller we could potentially call with a weak
hand just to try our luck. We also calculate whether or not the winnings are good
enough that we should play, if we might lose more than we could win, we fold.
For the decision the hand strength is what we rely on most because it tells us the
most about our hand\footnote{At the time of writing we are currently using the
	simple calculation because we had some problems with the more advanced.
	The advanced should be working now, but we have to little time to test
and so we have used the simple one}. Although with a more aggressive player this
can be changed.

For the betting decision we look at what is on the table and what we can potentially
win. We calculate an expected amount and bet according to this. If the player has a negative
amount of chips we only put in the amount needed to call so we don't end up with chip counts in the
billions.


\subsection{Phase 3}\label{phase 3 player}
In this phase we finally have all the bells and whistles to play poker. Combining
roll-out simulation, hand strength calculation and opponent modelling. Phase 3 does
not have the big advantage that phase 2 had over phase 1, but it has the potential
to become better than phase 1 and 2.

The big change is of course opponent modelling(\ref{opponent modelling}) which enables
this player to take a more informed decision about what the other players might be
doing.

As with phase 2 we just check the statistics when it comes to the pre flop betting round.
Since very little information is available to us and since using the opponent modelling
would be quite uncertain at this stage there is not much else to do. After the
flop however we can start using our arsenal to our advantage.

What we then do is
to look at our hand strength compared to the what the opponent modeller thinks is
the best other player at the table. We then combine the average strength from the
opponent modeller with the standard deviation\footnote{\url{http://mathworld.wolfram.com/StandardDeviation.html}}. To incorporate some aggressiveness we can apply different levels of confidence
in the weight of the standard deviation, meaning a conservative player might like
95\% confidence, but a more aggressive player might be good with only the average
or maybe even less.

 
\section{Results}\label{results}
Below is the results from the different phases of play. All results are 10 000 simulations
and for phase 3 there are and additional 1000 to see if the opponent modeller could learn
anything. Each player start with a 1000 chips and blinds are at 10 and 20 and increase.

\subsection{Phase 1}\label{phase 1 results}
\begin{tabular}{| l | c |}
	\hline
	Phase 1, Player 0, Aggressiveness: 0.0 & chip count: 63960280\\
	Phase 1, Player 1, Aggressiveness: 1.0 & chip count: -21112167\\
	Phase 1, Player 2, Aggressiveness: 2.0 & chip count: -11393325\\
	Phase 1, Player 3, Aggressiveness: -1.0 & chip count: -31450803\\
	\hline
\end{tabular}

\begin{tabular}{| l | c |}
	\hline
	Phase 1, Player 0, Aggressiveness: 0.0 & chip count: 62938731\\
	Phase 1, Player 1, Aggressiveness: 1.0 & chip count: -20831844\\
	Phase 1, Player 2, Aggressiveness: 2.0 & chip count: -11176039\\
	Phase 1, Player 3, Aggressiveness: -1.0 & chip count: -30926860\\
	\hline
\end{tabular}

\begin{tabular}{| l | c |}
	\hline
	Phase 1, Player 0, Aggressiveness: 0.0 & chip count: 63160394\\
	Phase 1, Player 1, Aggressiveness: 1.0 & chip count: -20883887\\
	Phase 1, Player 2, Aggressiveness: 2.0 & chip count: -11252784\\
	Phase 1, Player 3, Aggressiveness: -1.0 & chip count: -31019735\\
	\hline
\end{tabular}

In the above tables we have some a

\end{document}
